\documentclass[SE,lsstdraft,authoryear,toc]{lsstdoc}
% GENERATED FILE -- edit this in the Makefile
\newcommand{\lsstDocType}{SITCOMTN}
\newcommand{\lsstDocNum}{154}
\newcommand{\vcsRevision}{362d79f-dirty}
\newcommand{\vcsDate}{2025-06-20}


% Package imports go here.
<<<<<<< HEAD
=======
\usepackage{xcolor}
\usepackage{hyperref}
\usepackage{fancyvrb}
>>>>>>> d9cdfde (overleaf stamp jun 27 1600)

% Local commands go here.

%If you want glossaries
%\input{aglossary.tex}
%\makeglossaries

\title{Initial studies of photometric redshifts for ComCam data}

% This can write metadata into the PDF.
% Update keywords and author information as necessary.
\hypersetup{
    pdftitle={Photometric redshifts for ComCam data},
    pdfauthor={Melissa Graham},
    pdfkeywords={}
}

% Optional subtitle
% \setDocSubtitle{A subtitle}

\author{%
<<<<<<< HEAD
Melissa Graham
=======
  Eric Charles,
  John Franklin Crenshaw,
  Tianqing Zhang, 
  Sam Schmidt,
  Prakruth Adari, 
  Melissa Graham,  
>>>>>>> d9cdfde (overleaf stamp jun 27 1600)
}

\setDocRef{SITCOMTN-154}
\setDocUpstreamLocation{\url{https://github.com/lsst-sitcom/sitcomtn-154}}

\date{\vcsDate}

% Optional: name of the document's curator
% \setDocCurator{The Curator of this Document}

\setDocAbstract{%
This technote holds reports based on the first analyses of the Data Preview 1 (DP1) ComCam data by the Science Unit for photometric redshifts.  The Science Unit developed training and test datasets by matching DP1 data to high-quality reference redshifts obtained with spectroscopy, Grism data, and multi-band photometry.  
The Science Unit then used the RAIL software package to make photometric redshifts estimates using eight different algorithms, developed simple scientific performance metrics, used those metrics to explore how the performance of the algorithms varied with configuration changes, derived more optimized configurations of the algorithms and tested the performance of those configurations.   
}

% Change history defined here.
% Order: oldest first.
% Fields: VERSION, DATE, DESCRIPTION, OWNER NAME.
% See LPM-51 for version number policy.
\setDocChangeRecord{%
  \addtohist{1}{YYYY-MM-DD}{Unreleased.}{Melissa Graham}
}


\begin{document}

% Create the title page.
\maketitle
% Frequently for a technote we do not want a title page  uncomment this to remove the title page and changelog.
% use \mkshorttitle to remove the extra pages

% ADD CONTENT HERE
% You can also use the \input command to include several content files.

\appendix

% Include all the relevant bib files.
% https://lsst-texmf.lsst.io/lsstdoc.html#bibliographies
\section{References} \label{sec:bib}
\renewcommand{\refname}{} % Suppress default Bibliography section
\bibliography{local,lsst,lsst-dm,refs_ads,refs,books}

% Make sure lsst-texmf/bin/generateAcronyms.py is in your path
\section{Acronyms} \label{sec:acronyms}
\addtocounter{table}{-1}
\begin{longtable}{p{0.145\textwidth}p{0.8\textwidth}}\hline
\textbf{Acronym} & \textbf{Description}  \\\hline

DESI & Dark Energy Spectroscopic Instrument \\\hline
DP1 & Data Preview 1 \\\hline
DR1 & Data Release 1 \\\hline
ECDFS & Extended Chandra Deep Field-South Survey \\\hline
LSST & Legacy Survey of Space and Time (formerly Large Synoptic Survey Telescope) \\\hline
SE & System Engineering \\\hline
photo-z & photometric redshift \\\hline
\end{longtable}

% If you want glossary uncomment below -- comment out the two lines above
%\printglossaries





\end{document}
